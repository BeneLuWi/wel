\documentclass{scrartcl}

% Change font and reload config
\usepackage{helvet}
\renewcommand{\familydefault}{\sfdefault}

\usepackage{hyperref}

\usepackage{amsfonts,amsmath,amssymb,amsthm} 
\usepackage{marvosym}
\usepackage[ngerman]{babel}
\usepackage[utf8]{inputenc}
\usepackage{graphicx}

\title{Wireless Electric Lock\\ \small{Hardware-nahe Systemprogrammierung}}
\author{Aaron Winziers\\Benedikt Lüken-Winkels}
\begin{document}

\maketitle
\tableofcontents
\newpage

%===============
%
%  Section
%
%===============

\section{Hardware}\label{sec:hard}


%===============
%  Subsection
%===============
\subsection{WLAN-Modul}

\paragraph{Lua Module V2 ESP8266 ESP-12E}
\begin{itemize}
\item 4 MB Flash-Speicher
\item Programmierung in C
\item NodeMCU-Board
\item ESP-12E
\item Programmierung per Arduino möglich
\item Selbstinduziertes Schlafen und Aufwecken des Chips (Stromsparen)
\end{itemize}


%===============
%  Subsection
%===============

\subsection{Vorteile gegenüber Arduino}

\begin{itemize}
\item Höhere Taktfrequenz (80-160 Mhz)
\item Vielfältige Speicherressourcen
\end{itemize}




\newpage

%===============
%
%  Section
%
%===============

\section{Programmierung}\label{sec:soft}


%===============
%  Subsection
%===============

\subsection{Messen des Klingelrythmusses}\label{ssec:mess}

\paragraph{Ideen}
\begin{itemize}
\item Unterscheidung der verschiedenen Klingellängen durch Anpassung an den Klingler
\begin{itemize}
\item Datenerhebung durch verschiedene Testpersonen, wie sich die Klingellängen bei lang und kurz unterscheiden
\item Bsp. $Time($Kurz$) = \frac{2}{3} Time($Lang$)$
\item[$\Rightarrow$] Dafür entweder Heuristik oder einmal lang, einmal kurz zum justieren.
\end{itemize}
\item Orientierung für den Klingler durch visuelle Elemente
\begin{itemize}
\item Display, das die Klingellänge durch Balken anzeigt \ref{ssec:lcd}
\item LED, das ein Tempo angibt
\end{itemize}
\end{itemize}


%===============
%  Subsection
%===============

\subsection{1602A (HD44780) LCD - Anzeige}\label{ssec:lcd} 

% Spezifikation
% https://www.openhacks.com/uploadsproductos/eone-1602a1.pdf

5 Pixel breit und 8 Pixel hoch. Zwischen den Feldern sind Lücken, die 1 Pixel groß sind.
\paragraph{Idee} Darstellung der Klingellänge als horizontal aufsteigende Balken
\begin{itemize}
\item Probleme bei der Darstellung
\begin{itemize}
\item Dynamische Aktualisierung der Pixel
\item Custom Characters (keine Standardbelegung)
\end{itemize}
\end{itemize}

%===============
%  Subsection
%===============

\subsubsection{Verkabelung}

% Tutorial 
% https://www.instructables.com/id/Step-By-Step-LCD-wiring-4-Bit-Mode-and-Programmi/

Um die Kommunikation zwischen Arduino und dem LCD zu gewährleisten musste das Breadboard auch entsprechend verkabelt werden. Dazu musste ein Weiteres Powersupply angesteckt werden da der Arduino nur 3V3 Output hat und 5V gebraucht sind um das LCD anzusteuern.

%===============
%  Subsection
%===============

\subsubsection{LiquidCristral Library}

%===============
%
%  END
%
%===============

\end{document}